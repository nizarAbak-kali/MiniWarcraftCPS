\part{Organisation du Travail}

\section{Organisation du Projet}
Le projet est à  faire en binôme . Le projet ce compose de trois dossier : un dossier "src" qui contient les sources du projets , un dossier "spécifications"  qui contient les spécifications des services , et un dossier "doc" qui contient le rapport ainsi que le sujet du projet.\\
 Le dossier contenant les sources se découpe en les sous-dossier suivants : 
 	\begin{itemize}
 	\item services : qui contient les services coder en java .
 	\item implementations : contient les implémentations  des services .
 	\item decorators : contient les décorateurs implémentant les services .
 	\item contracts : contient les contrats héritant des décorateurs .
 	\item test : contient deux mains , un qui lance programme avec les contrats et un sans.
 	\item exceptions : contient les classes qui gèrent les exceptions du programme.
 	\end{itemize}
 	
 	Lors de la programmation nous nous y sommes pris par "étage" . C'est à dire, que nous avons d'abord écrit tout les services, puis  nous somme passer par par l’implémentation de ces services, ainsi de suites suivant l'ordre décrit ci-dessus .
 	

\section{Outils utilisés}

\subsection{Java et JUnit}
Nous avons programmer le jeu en Java et cela pour plusieurs raisons. Premièrement , les deux membres du binômes connaissent et maîtrise le langage.Deuxièmement,  nous avons été initié à la programmation par contrat en Java lors de nos travaux pratique. finalement, Java possèdes la librairie de teste JUnit très utiles pour pratiqué des testes unitaires .

\subsection{Communication et gestion de projet}
La communication au sain du groupe c'est principalement faite par mail et téléphone . La gestion du projet a été faite par Github. Ce choix et surtout dues au fait que l'outils qu'est Github est extrement pratique pour le travail en équipe et pour le télétravail .


