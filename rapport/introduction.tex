\part{Introduction}

\section{Enoncé}
	Ce projet consiste à développer un modèle, simplifié, avec
test embarqué du mode multi-joueur du jeu Warcraft:
Orcs and Humans TM , produit par Blizzard en 1994.
Dans cet épisode de la série, comme dans tout épisode de
cette série à l’innovation discutable, les joueurs dispensent
des ordres à gogo à leur chers sujets  - orcs ou humains -
en appuyant et déplaçant furieusement leur main sur un
objet rond à boutons cliquables...

\section{Objectif du projet}
L’objectif du projet est multiple :
	\begin{itemize}
		\item[-] donner une spécification complète du modèle de jeu dans le langage de spécification vu en cours
		\item[-] réaliser à partir de la spécification une implémentation contractualisée, service et contrat sous
forme de tests embarqués, en utilisant le langage de votre choix comme par exemple le Java vu en
TME
		\item[-] définir à partir de la spécification des objectifs de test pour assurer les couvertures logiques, les paires de transitions, et une suite de scénarios utilisateurs
		\item[-]	réaliser à partir des objectifs de test une implémentation des cas de tests spécifiés, en Junit par
exemple
		\item[-] réaliser deux versions complètes du jeu, dont une totalement “buggée”. Ici, peu importe si les deux
réalisations sont avec interface graphique ou non. En revanche, elles doivent être en dehors de tout
modèle du jeu : pas de spécification, pas de contrats, pas de tests MBT
		 \item[-] recueillir les messages d’erreur lors de la confrontation de la version buggée avec les tests spécifiés.
	\end{itemize}

